\documentclass{beamer}

\usetheme{Dresden}
\usecolortheme{spruce}

\usepackage{makecell}
\usepackage{tikz}
\usetikzlibrary{automata, arrows.meta, positioning}

\definecolor{darkGreen}{RGB}{0,102,51}

\setbeamerfont{block body}{size= \small}
\setbeamercolor{block title}{fg=white,bg=darkGreen!100!white}
\setbeamercolor{block body}{bg=darkGreen!70!black,bg=white!60!darkGreen}
\setbeamercolor{block title}{fg=white,bg=darkGreen!100!white}
\setbeamercolor{block body}{bg=darkGreen!70!black,bg=white!60!darkGreen}
\setbeamerfont{block body}{size= \small}
\setbeamercolor{enumerate item}{bg=black, fg=black}

\title[ ]{Lab 4: CS 699 Presentation}
\author{Sagar Poudel}
\institute{IIT Bombay}
\date{\today}

\begin{document}
\frame{\titlepage}

\begin{frame}{Topics to cover}
    \tableofcontents
\end{frame}

\section{Paragraph}

\begin{frame}{Statements}
    
    \begin{block}{Pumping Lemma for Regular Languages$^{1}$}
       \emph{ $
        \text{For any regular language L, there exists an integer n, such that for all x} \in \text{L with } |x| \geq \text{n, there exists u, v, w} \in \Sigma^{*} \text{, such that x = uvw, and }$\\
        
        \emph{(1)} $ \left|uv\right| \leq $ n\\
        \emph{(2)} $\left|v\right| \geq 1 $\\
        \emph{(3)} $\text{for all i } \geq 0; uv^iw\ \epsilon{} $ L}
    \end{block}
\end{frame}

\section{}
\begin{frame}
    \frametitle{Statements}
    \begin{block}{Pumping lemma for Regular Languages$^{1}$}
    \emph{ $\text{For any regular language L, there exists an integer n, such that for all x} \in \text{L with } |x| \geq \text{n, there exists u, v, w} \in \Sigma^{*} \text{, such that x = uvw, and }$\\
    
         \emph{(1)} $ \left|uv\right| \leq $ n\\
        \emph{(2)} $\left|v\right| \geq 1 $\\
        \emph{(3)} $\text{for all i } \geq 0; uv^iw\ \epsilon{} $ L}
    \end{block}

    \begin{block}{Rice Theorem}
        \textit{It states that any non trivial sementic property of a language which is required by a Turing machine is undecidable. A property P, is the language of all turing machines that satisfy that property.}
    \end{block}
\end{frame}

\section{Column}

\begin{frame}{Classification}

    \begin{table}[]
        \begin{tabular}{l l l}
        \textbf{Grammer}& \textbf{Language}& \textbf{Automata}\\[2pt]
        \hspace{2mm} \textcolor{blue}{1.}  RG&\hspace{2mm}\textcolor{blue}{1.}  Regular Language &\hspace{2mm}\textcolor{blue}{1.}  FA  \\
        \hspace{2mm} \textcolor{blue}{2.}  CFG&\hspace{2mm}\textcolor{blue}{2.}  Context Free Language &\hspace{2mm}\textcolor{blue}{2.}  PDA   \\
        \hspace{2mm} \textcolor{blue}{3.}  UG&\hspace{2mm}\textcolor{blue}{3.}  Recursively Enumerable Sets &\hspace{2mm}\textcolor{blue}{3.}  TM 
         \end{tabular}
    \end{table}
\end{frame}


\section{Flowchart}

\begin{frame}{Automata}
    \tikzstyle{states} = [state, minimum width=1cm, text centered, draw=black]
    \begin{figure}
        \centering
        \begin{tikzpicture} [node distance = 2.7cm, on grid,every initial by arrow/.style={->}]
        \node (q1) [states,accepting, initial, initial distance = 1.2cm, inner sep = 1pt, initial text=\textbf{start}]
        {$q1$};
        \node (q2) [states ,right = of q1] {q2};
        \node (q3) [states ,below = of q2] {q3};
        \node (q4) [states ,below = of q1] {q4};
        \path [-stealth, thick]
        (q1) edge[bend left, -Latex] node[red, pos=0.5, fill=white] {$0$}   (q2)
        (q2) edge[bend left, -Latex] node[red, pos=0.5, fill=white] {$0$}   (q1)
        (q1) edge[bend left, -Latex] node[red, pos=0.5, fill=white] {$1$}   (q4)
        (q4) edge[bend left, -Latex] node[red, pos=0.5, fill=white] {$1$}   (q1)
        (q2) edge[bend left, -Latex] node[red, pos=0.5, fill=white] {$1$}   (q3)
        (q3) edge[bend left, -Latex] node[red, pos=0.5, fill=white] {$1$}   (q2)
        (q4) edge[bend left, -Latex] node[red, pos=0.5, fill=white] {$0$}   (q3)
        (q3) edge[bend left, -Latex] node[red, pos=0.5, fill=white] {$0$}   (q4);
        \end{tikzpicture}
        \caption{Automata accepting even 0’s and 1’s}
    \end{figure}
\end{frame}


\section{Table}

\begin{frame}{State Transition Table}
    \begin{table}[]
        \centering
    \begin{tabular}{c c c}
        \Xhline{4\arrayrulewidth}
        \textbf{STATES} & \textbf{0's} & \textbf{1's}\\
        \hline
        \textbf{q1} &q2 & q4 \\
        q2 & \textbf{q1} & q3   \\
        q3 & q4 & q2\\
        q4 & q3 & \textbf{q1} \\
        \Xhline{4\arrayrulewidth}
     \end{tabular}
      \caption{State Table for above Automata}
        \label{tab:my_label}
    \end{table}
\end{frame}


\section{}
\begin{frame}{}
  \centering \Huge
  {Thank You!}
\end{frame}
\end{document}
